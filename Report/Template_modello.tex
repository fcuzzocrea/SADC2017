\documentclass[11pt,a4paper]{report}
\usepackage[scale=0.75]{geometry}
\usepackage[utf8]{inputenc}
\usepackage{microtype}
\usepackage[hyphens]{url}
\usepackage[unicode=true,bookmarks=true,bookmarksnumbered=false,bookmarksopen=false,breaklinks=false,pdfborder={0 0 1},backref=false,colorlinks=false]{hyperref}
\usepackage{graphicx}
\usepackage{datetime}
\usepackage{fancyhdr}
\pagestyle{fancy}
\setcounter{secnumdepth}{1}
\setcounter{tocdepth}{3}
\setlength{\parskip}{\smallskipamount}
\setlength{\parindent}{0pt}
\usepackage[english]{babel}
\usepackage{float}
 
\makeatletter
%%%%%%%%%%%%%%%%%%%%%%%%%%%%%% Textclass specific LaTeX commands.
\newlength{\lyxlabelwidth}      % auxiliary length 

%%%%%%%%%%%%%%%%%%%%%%%%%%%%%% User specified LaTeX commands.
% Customization file for the titlepage and document
%************************************************************
% Required stuff
%************************************************************
\usepackage[T1]{fontenc}
\usepackage[english]{babel}
\usepackage{graphicx}
\usepackage{euler}
\usepackage[detect-all]{siunitx}
\usepackage{sectsty}
\usepackage[font={footnotesize }]{caption}
\usepackage{multicol}
\usepackage{prettyref}
\usepackage[scale=0.75]{geometry}
\usepackage[utf8]{inputenc}
\usepackage{microtype}
\usepackage[hyphens]{url}
\usepackage[unicode=true,bookmarks=true,bookmarksnumbered=false,bookmarksopen=false,breaklinks=false,pdfborder={0 0 1},backref=false,colorlinks=false]{hyperref}
\usepackage{datetime}
\usepackage{float}
\usepackage[nointegrals]{wasysym}
\usepackage{amsmath}
%Bibliography
\usepackage{etoolbox}
% Acronyms
\usepackage[acronym,nonumberlist,nopostdot,section=section,toc,numberedsection]{glossaries}

\allsectionsfont{\rmfamily}

% Page customization
\usepackage{fancyhdr}
\pagestyle{fancy}

% Color
\usepackage{color}
\definecolor{light-gray}{gray}{0.85}
\definecolor{dark-gray}{gray}{0.75}

\fancyhead{}  % clear all header fields
\fancyhead[LO,RE]{\rule[-2ex]{0pt}{2ex}\fontsize{9}{11} \selectfont \myPhase : \myTitle}
\fancyhead[CO,C]{\fontsize{9}{11} \selectfont \myIPT}
\fancyfoot{}  % clear all footer fields
\fancyfoot[RO,LE]{\fontsize{6}{11} \selectfont \includegraphics[height=0.2cm]{gfx/CC} This work is licensed under a Creative Commons Attribution-ShareAlike 4.0 International License.}
\fancyheadoffset[L,R]{0.2pt}
\renewcommand{\headrulewidth}{0.2pt}
\renewcommand{\footrulewidth}{0.2pt}
\renewcommand{\headrule}{\hbox to\headwidth{%
   \leaders\hrule height \headrulewidth\hfill}}
\renewcommand{\footrule}{\hbox to\headwidth{%
    \leaders\hrule height \headrulewidth\hfill}}
\hypersetup{colorlinks=true, linkcolor=blue ,linktoc=page,citecolor=black}

%************************************************************
% Redefining numbering for sections
%************************************************************
\renewcommand*\thesection{\arabic{section}}

%************************************************************
% Defining useful commands
%************************************************************
\newcommand{\norm}[1]{\left\lVert#1\right\rVert}

%************************************************************
% Cross reference set-up
%************************************************************
\newrefformat{tab}{Table\,\ref{#1}}
\newrefformat{fig}{Figure\,\ref{#1}}
\newrefformat{eq}{Eq.\,\textup{(\ref{#1})}}
\newrefformat{sec}{Sec.\,\ref{#1}}
\newrefformat{sub}{Sec.\,\ref{#1}}

%************************************************************
% Fancy stuff
%************************************************************
\newcommand{\titlecap}[1]{\Huge{\textrm{#1}}}
\newcommand{\subtitlecap}[1]{\Large{\textsc{#1}}}
\newcommand{\sscap}[1]{\textbf{#1}}
\newcommand{\strong}[1]{\textbf{#1}}
\setlength{\headheight}{60pt} %%or

%************************************************************
% Helpful stuff to modify here, not in the LyX Document
%************************************************************
\newcommand{\myDate}{\today}
\newcommand{\myGroup}{}
\newcommand{\myUrl}{\url{https://github.com/fcuzzocrea/SADC2017}}
\newcommand{\myUni}{}
\newcommand{\myPhase}{Spacecraft Attitude Dynamics and Control}
\newcommand{\myProject}{}
\newcommand{\myIPT}{}
\newcommand{\myTitle}{Assignment Report}
\newcommand{\myAuthor}{Francescodario Cuzzocrea}
\newcommand{\myEmail}{francescodario.cuzzocrea@mail.polimi.it}

\newcommand{\mail}[1]{\href{mailto:#1}{\texttt{#1}}}

\setlength{\textfloatsep}{\baselineskip}

\makeatother

\begin{document}
\input{Titlepage.tex}

%*******************************************************
% Titleback
%*******************************************************
\thispagestyle{empty}

\hfill
\vspace{5cm}
\strong{ }\\
The following report will contain the procedure and the mathematical models used to simulate the behaviour a 6U Cubesat in a LEO orbit.
The sensor installed on the cubesat are a tri-axial magnetometer and a tri-axial gyroscope. 
The actuators installed on the cubesat to control the attitude are 3 magnetic torquers and a reaction wheel.

The latest version of the simulator can be found at : https://github.com/fcuzzocrea/SADC2017

\vfill

\begin{multicols}{2}
\medskip
\noindent{\sscap{Website}}: \\
\url{https://github.com/fcuzzocrea/SADC2017}

\end{multicols}
\vspace{1cm}
\hrule
\bigskip
\clearpage


\pagenumbering{roman}

\tableofcontents{}

\clearpage{}

\pagenumbering{arabic}

\setcounter{page}{3}

\chapter{Parameters and Mission Requirements}
In this section the parameters used to simulate the behavior of the assigned Spacecraft will be defined, along with the rationale for their selection. 
The parameters are divided in four main categories : 

\begin{itemize}
 \item Orbital Parameters
 \item Sensors Parameters
 \item Actuators Parameters
 \item Enviromental Parameters
 \item Mission Requirements
\end{itemize}

\section{Orbital Parameters}

Only the altitude of the orbit has be assigned by the customer. 
The other orbital parameters has been chosen as follows : 
\begin{center}
	\renewcommand{\arraystretch}{1.2}
	\begin{tabular}{|c|c|c|}
		\hline
		Orbital Parameters & Symbols & Values \\
		\hline
		Altitude & h & \textit{600 [Km]} \\
		\hline
		Orbit radius & r & \textit{6917 [Km]}\\
		\hline
		Semimajor axis & a & \textit{6917 [Km]}\\
		\hline
		Inclination & i & \textit{0 $^{o}$}\\
		\hline
		Pericenter anomaly & $\omega$ & \textit{0 $^{o}$}\\
		\hline
		True anomaly & $\theta$ &  \textit{0 $^{o}$}\\
		\hline
		Eccentricity & e &  \textit{0}\\
		\hline	
	\end{tabular}
\end{center}

The selected orbit is circular with $0\ ^{o}$ inclination, meaning an equatorial orbit. Starting from this parameters the period of the orbit is computed and a multiple of its value is adopted as the simulation time in order to run the model.

$$T_{orbit} = 2\pi\sqrt{\frac{a^{3}}{mu}}$$

\section{Sensors Parameters}
\section{Actuators Parameters}
\section{Enviromental Parameters}
\section{Mission Requirements}

\chapter{Spacecraft Description}

\section{General Description}
\section{Sensors Description}
\subsection{Magnetometer}
\subsection{Gyroscope}
\section{Actuators Description}
\subsection{Magnetic Torquers}
\subsection{Reaction Wheels}
\section{Onboard Computer}

\chapter{ADCS Architecture}

Schema

\chapter{Mathematical Modeling}

\section{Dynamics}
\subsection{Euler Equations}
\subsection{Enviromental Disturbances}
\subsubsection{Gravity Gradient}
\subsubsection{Earth's Magnetic Field}
\subsubsection{Solar Radiation Pressure}
\subsubsection{Aerodynamic Drag}
\section{Kinematics}
\subsection{Euler Angles}
\subsection{Quaternions}
\section{Sensors Models}
\subsection{Magnetometer}
\subsection{Gyroscope}
\section{Actuators Models}
\subsection{Magnetic Torquers}
\subsection{Reaction Wheels}

\chapter{Control}
\section{Quaternion Feedback Contrl}
\section{Mixer Matrix}

\chapter{Results}

\chapter{Conclusions}

\newpage
\end{document}
